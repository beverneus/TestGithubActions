\documentclass[10pt,a4paper,twoside]{article}
\usepackage[dutch]{babel}
%laad de pakketten nodig om wiskunde weer te geven :
\usepackage{amsmath,amssymb,amsfonts,textcomp}
%laad de pakketten voor figuren :
\usepackage{graphicx}
\usepackage{float,flafter}
\usepackage{hyperref}
\usepackage{inputenc}
%pakketen voor mooiere tabellen:
\usepackage{hhline}
\usepackage{multirow}
%zet de bladspiegel :
\setlength\paperwidth{20.999cm}\setlength\paperheight{29.699cm}\setlength\voffset{-1in}\setlength\hoffset{-1in}\setlength\topmargin{1.499cm}\setlength\headheight{12pt}\setlength\headsep{0cm}\setlength\footskip{1.131cm}\setlength\textheight{25cm}\setlength\oddsidemargin{2.499cm}\setlength\textwidth{15.999cm}

\begin{document}
\begin{center}
\hrule

\vspace{.4cm}
{\bf {\Huge DIY Practicum:\\ De Archimedes Kracht}}
\vspace{.2cm}
\end{center}
{\bf Quinten De Leenheer, Liam Depamelaere, Lars De Volder, Matse Vandewalle}  \\
{\bf Eerste Bachelor Fysica en Sterrenkunde} \hspace{\fill} \\
11 april 2024 \\
\hrule

\section{Inleiding}

\section{Experimentele methode}

\section{Meetresultaten en Bespreking}

De meetresultaten bevinden zich in tabellen \ref{Tabel1}, \ref{Tabel2}, en \ref{Tabel3}.
Per $250ml \pm Xml$ toegevoegd stijgt het oppervlak $2,8cm \pm 0,1cm$,
hieruit valt de oppervlakte van het vloeistof oppervlak als volgt te berekenen:

\begin{align}
	&\text{De kan is een cilinder,
		  er wordt een deel van de cilinder met hoogte 2,8 cm gebruikt}\\
	V_{cil} &= r^2 \pi \cdot h\\
	r &= \sqrt{\frac{V_{cil}}{h \cdot \pi}}\\
	r &= \sqrt{\frac{250cm^3}{2,8cm \cdot \pi}}\\
	  &= 5,3310905cm \emph{   tijdelijk extra decimalen}
\end{align}

Met fout:
\begin{align}
	AF(r) &= 
\end{align}

\begin{table}
	\begin{center}
		\caption{Meetwaarden bij water}
		\label{Tabel1}
		\begin{tabular}{|c|c|c|}
			\hline
			Massa voorwerp [g] & Verandering in hoogte [cm] & Volume ondergedompeld [$m^3$]\\
			\hline
			299 $\pm$ 1 & 3.3 $\pm$ 0.1 & ntb\\
			\hline
			316 $\pm$ 1 & 3.4 $\pm$ 0.1 & ntb\\
			\hline
			338 $\pm$ 1 & 3.6 $\pm$ 0.1 & ntb\\
			\hline
			356 $\pm$ 1 & 3.8 $\pm$ 0.1 & ntb\\
			\hline
			376 $\pm$ 1 & 4.0 $\pm$ 0.1 & ntb\\
			\hline
			398 $\pm$ 1 & 4.2 $\pm$ 0.1 & ntb\\
			\hline
			418 $\pm$ 1 & 4.4 $\pm$ 0.1 & ntb\\
			\hline
			436 $\pm$ 1 & 4.6 $\pm$ 0.1 & ntb\\
			\hline
			462 $\pm$ 1 & 4.9 $\pm$ 0.1 & ntb\\
			\hline
			486 $\pm$ 1 & 5.1 $\pm$ 0.1 & ntb\\
			\hline
		\end{tabular}
	\end{center}
\end{table}

\begin{table}
	\begin{center}
		\caption{Meetwaarden bij ethanol}
		\label{Tabel2}
		\begin{tabular}{|c|c|c|}
			\hline
			Massa voorwerp [g] & Verandering in hoogte [cm] & Volume ondergedompeld [$m^3$]\\
			\hline
			299 $\pm$ 1 & 3.9 $\pm$ 0.1 & ntb\\
			\hline
			321 $\pm$ 1 & 4.2 $\pm$ 0.1 & ntb\\
			\hline
			348 $\pm$ 1 & 4.5 $\pm$ 0.1 & ntb\\
			\hline
			366 $\pm$ 1 & 4.7 $\pm$ 0.1 & ntb\\
			\hline
			385 $\pm$ 1 & 5.0 $\pm$ 0.1 & ntb\\
			\hline
			403 $\pm$ 1 & 5.2 $\pm$ 0.1 & ntb\\
			\hline
			423 $\pm$ 1 & 5.4 $\pm$ 0.1 & ntb\\
			\hline
		\end{tabular}
	\end{center}
\end{table}

\begin{table}
	\begin{center}
		\caption{Meetwaarden bij zonnebloemolie}
		\label{Tabel3}
		\begin{tabular}{|c|c|c|}
			\hline
			Massa voorwerp [g] & Verandering in hoogte [cm] & Volume ondergedompeld [$m^3$]\\
			\hline
			299 $\pm$ 1 & 4.0 $\pm$ 0.1 & ntb\\
			\hline
			316 $\pm$ 1 & 3.6 $\pm$ 0.1 & ntb\\
			\hline
			333 $\pm$ 1 & 3.8 $\pm$ 0.1 & ntb\\
			\hline
			351 $\pm$ 1 & 4.0 $\pm$ 0.1 & ntb\\
			\hline
			371 $\pm$ 1 & 4.3 $\pm$ 0.1 & ntb\\
			\hline
			400 $\pm$ 1 & 4.5 $\pm$ 0.1 & ntb\\
			\hline
			423 $\pm$ 1 & 4.7 $\pm$ 0.1 & ntb\\
			\hline
			446 $\pm$ 1 & 4.9 $\pm$ 0.1 & ntb\\
			\hline
			468 $\pm$ 1 & 5.2 $\pm$ 0.1 & ntb\\
			\hline
			486 $\pm$ 1 & 5.3 $\pm$ 0.1 & ntb\\
			\hline
		\end{tabular}
	\end{center}
\end{table}

\section{Besluit}

\end{document}
